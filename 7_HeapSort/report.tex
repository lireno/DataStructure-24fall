\documentclass[UTF8]{ctexart}
\usepackage{geometry, CJKutf8}
\usepackage[affil-it]{authblk}
\usepackage{hyperref}
\geometry{margin=1.5cm, vmargin={0pt,1cm}}
\setlength{\topmargin}{-1cm}
\setlength{\paperheight}{29.7cm}
\setlength{\textheight}{25.3cm}

% useful packages.
\usepackage{amsfonts}
\usepackage{amsmath}
\usepackage{amssymb}
\usepackage{amsthm}
\usepackage{enumerate}
\usepackage{graphicx}
\usepackage{multicol}
\usepackage{fancyhdr}
\usepackage{layout}
\usepackage{listings}
\usepackage{float, caption}

\lstset{
    basicstyle=\ttfamily, basewidth=0.5em
}

\begin{document}

\pagestyle{fancy}
\fancyhead{}
\lhead{梁育玮, 3230102923}
\chead{数据结构与算法第七次作业}
\rhead{\today}
\title{数据结构与算法第七次作业}

\author{梁育玮 3230102923
  \thanks{Electronic address: \texttt{liangyuwei631@gmail.com}}}
\affil{(Mathematics and Applied Mathematics 2302), Zhejiang University }


\date{\today}

\maketitle

\section{HeapSort设计思路}

HeapSort算法的设计思路如下:

\begin{enumerate}
    \item \textbf{构建最大堆}:
    \begin{itemize}
        \item 从数组的中间位置开始,逐步向前遍历每个元素,并对每个元素进行堆化操作。
        \item 堆化操作的目的是确保当前节点及其子树满足最大堆的性质,即父节点的值大于或等于其子节点的值。
    \end{itemize}
    \item \textbf{排序过程}:
    \begin{itemize}
        \item 将堆顶元素(即最大值)与数组的最后一个元素交换位置,这样最大值就被移到了数组的末尾。
        \item 然后对剩余的元素(不包括已经排序的部分)进行堆化操作,以重新构建最大堆。
        \item 重复上述步骤,直到所有元素都被排序。
    \end{itemize}
\end{enumerate}
下面是伪代码

\begin{lstlisting}
    Heapify(A, n, i)
        l = 2 * i + 1
        r = 2 * i + 2
        largest = i
        if l <= A.heapSize and A[l] > A[largest]
            largest = l
        if r <= A.heapSize and A[r] > A[largest]
            largest = r
        if largest != i
            swap A[i] and A[largest]
            Heapify(A, largest)

    BuildHeap(A)
    A.heapSize = A.length
    for i = A.length / 2 downto 0
        Heapify(A, 0)

    HeapSort(A)
        BuildHeap(A)
        for i = A.length downto 1
            swap A[1] and A[i]
            Heapify(A, i, 0)
\end{lstlisting}

\section{测试结果}
\subsection{结果展示}
下面是HeapSort算法的测试结果(1000000个数据),实验环境为 -std=c++17 O2:
\begin{table}[H]
    \centering
    \begin{tabular}{|c|c|c|}
        \hline
        Sequence Type & HeapSort Time (seconds) & std::sort\_heap Time (seconds) \\
        \hline
        Random Sequence & 0.0891352 & 0.0967212 \\
        \hline
        Ordered Sequence & 0.0398378 & 0.0588818 \\
        \hline
        Reverse Sequence & 0.0419689 & 0.0743362 \\
        \hline
        Repeated Sequence & 0.0746369 & 0.224179 \\
        \hline
    \end{tabular}
    \caption{Performance comparison between HeapSort and std::sort\_heap}
    \label{tab:performance}
\end{table}

从表\ref{tab:performance}中可以看出,HeapSort在所有测试的序列类型上都表现出了比std::sort\_heap更好的性能。这表明HeapSort在处理不同类型的序列时具有较高的效率。

\subsection{原因分析}
std::sort\_heap在程序运行前会对序列进行一系列检查,会抛出异常,而HeapSort没有这一步。

下面是std::sort\_heap函数开始时的一些检查,例如确保可以比较,比较无自反性,是一个有效的堆等等。
\begin{lstlisting}
__glibcxx_function_requires(_Mutable_RandomAccessIteratorConcept<_RandomAccessIterator>)
__glibcxx_requires_valid_range(__first, __last);
__glibcxx_requires_irreflexive(__first, __last);
__glibcxx_requires_heap(__first, __last);
\end{lstlisting}

同时,std::sort\_heap调用了多个内部函数(如 std::\_\_pop\_heap 和 std::\_\_adjust\_heap),每个函数都进行了多层封装和抽象。这些抽象在编译时会增加复杂性,并可能降低运行时性能。
\end{document}
