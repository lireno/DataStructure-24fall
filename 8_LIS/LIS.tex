\documentclass[a4paper]{article}
\usepackage[affil-it]{authblk}

\usepackage{listings}
\usepackage[UTF8]{ctex}
\usepackage{amsmath}
\usepackage{amssymb}
\usepackage{geometry}
\usepackage{graphicx}

\title{LIS(单调递增数列)实现}

\author{梁育玮 3230102923}

\date{\today}

\begin{document}
\maketitle
\subsection*{I. 大致思路}
维护一个大小不超过$n$的一维数组$s$和他的大小size\_以及一个数组b,$s[i]$表示长度为$i$的递增数列里面,末值的最小可能,保证$s$是升序的。$b[i]$表示第$i$个元素结尾的递增序列最大长度。

每读入一个数$a[i]$,记$j$是满足$s[j]<a[i]$的最大下标,将$s[j+1]$设为$t$,$b[i]$设为j+1,同时视情况更新size的值,最终数组s的长度即为LIS序列的最大长度

最后通过数组b倒着找满足要求的序列

\subsection*{伪代码}
下面是伪代码
\begin{lstlisting}
a is a given array
n = a.size
b = an array of size n
s = an array of size n+1
size_ = 1
s[1] = a[0]
b[0] = 1
for i = 1 to n-1
    //bijection search
    pos = 0
    l, r = 1, size
    while l <= r
        mid = (l + r)/2
        if a[i] > s[mid]
            pos = mid
            l = mid + 1
        else r = mid - 1
    s[pos + 1] = a[i]
    b[i] = pos + 1
m, j = size_, n-1
LST = an array of size size_
//print the LIS
while m > 0
    while b[j] != m
        j = j -1
    LST[m-1] = a[j]
    m = m - 1
\end{lstlisting}
同文件夹下的'1.py'中有根据伪代码自动生成的python代码,python代码中有对接下来展示的数据的详细处理

\subsection*{II. 例子}
给定一列数据:a = [2, 1, 4, 3, 2, 5, 6, 3, 2] 
\begin{itemize}
    \item 处理2:将s[1]设为2,size设为1,b[0] = 1
    \item 处理1:没有比1小的,将s[1]设为1,b[1] = 1
    \item 处理4:1比4小,将s[2]设为4,同时更新size\_为2,b[2] = 2
    \item 处理3:1比3小,将s[2]设为3,b[3] = 2
    \item 处理2:1比2小,将s[2]设为2,b[4] = 2
    \item 处理5:2比5小,将s[3]设为5,同时更新size\_为3,b[5] = 3
    \item 处理6:5比6小,将s[4]设为5,同时更新size\_为4,b[6] = 4
    \item 处理3:2比3小,将s[3]设为3,b[7] = 3
    \item 处理2:1比2小,将s[2]设为2(实际上没有变化),b[8] = 2
    \item size = 4为数组长度,b=[1, 1, 2, 2, 2, 3, 4, 3, 2]
    \item 最后找到要求的LIS
\end{itemize}


\subsection*{时间复杂度分析}
二分查找的时间复杂度为\(\Theta\)(log size\_),故构建数组s和b的时间复杂度是\(\mathcal{O}(n \log n )\),
构建LST的时间复杂度为\(\mathcal{O}(n)\),故整体时间复杂度为\(\mathcal{O}(n \log n)\)


\end{document}