\documentclass[UTF8]{ctexart}
\usepackage{geometry, CJKutf8}
\usepackage[affil-it]{authblk}
\usepackage{hyperref}
\geometry{margin=1.5cm, vmargin={0pt,1cm}}
\setlength{\topmargin}{-1cm}
\setlength{\paperheight}{29.7cm}
\setlength{\textheight}{25.3cm}

% useful packages.
\usepackage{amsfonts}
\usepackage{amsmath}
\usepackage{amssymb}
\usepackage{amsthm}
\usepackage{enumerate}
\usepackage{graphicx}
\usepackage{multicol}
\usepackage{fancyhdr}
\usepackage{layout}
\usepackage{listings}
\usepackage{float, caption}

\lstset{
    basicstyle=\ttfamily, basewidth=0.5em
}

\begin{document}

\pagestyle{fancy}
\fancyhead{}
\lhead{梁育玮, 3230102923}
\chead{数据结构与算法第六次作业}
\rhead{\today}
\title{数据结构与算法第六次作业}

\author{梁育玮 3230102923
  \thanks{Electronic address: \texttt{liangyuwei631@gmail.com}}}
\affil{(Mathematics and Applied Mathematics 2302), Zhejiang University }


\date{\today}

\maketitle

\section{\texttt{remove} 函数的实现}
先迭代寻找要删的节点,如果碰到有两个子节点的情况,使用`detachMin`函数找到右子树的最小值,将最小值从右子树分离出来,替换要删的节点。其它情况都很好操作。

由于`remove`函数实现为上节课内容,而且没有什么花样,这里就不再赘述了。接下来讲一下如何在`remove`函数运行过程中调用balance,实现平衡二叉树。

其实也很简单,在`remove`函数的最后调用`balance`就可以了,这样子就能迭代使用时间。另外,我维护了一个变量`is\_removed', 同时将remove函数的返回值变为bool型,表示是否找到并删除了节点。如果删除了节点,就将`is\_removed`置为true,并递归调用`balance`函数。

通过以上方法,我成功得将程序的平均时间控制在了2200ms左右。

blance函数的实现都是抄学在浙大上code里面的代码(即书上代码)。
\end{document}

%%% Local Variables: 
%%% mode: latex
%%% TeX-master: t
%%% End: 