\documentclass[UTF8]{ctexart}
\usepackage{geometry, CJKutf8}
\usepackage[affil-it]{authblk}
\geometry{margin=1.5cm, vmargin={0pt,1cm}}
\setlength{\topmargin}{-1cm}
\setlength{\paperheight}{29.7cm}
\setlength{\textheight}{25.3cm}

% useful packages.
\usepackage{amsfonts}
\usepackage{amsmath}
\usepackage{amssymb}
\usepackage{amsthm}
\usepackage{enumerate}
\usepackage{graphicx}
\usepackage{multicol}
\usepackage{fancyhdr}
\usepackage{layout}
\usepackage{listings}
\usepackage{float, caption}

\lstset{
    basicstyle=\ttfamily, basewidth=0.5em
}

\begin{document}

\pagestyle{fancy}
\fancyhead{}
\lhead{梁育玮, 3230102923}
\chead{数据结构与算法第四次作业}
\rhead{Oct.16th, 2024}
\title{数据结构与算法第四次作业}

\author{梁育玮 3230102923
  \thanks{Electronic address: \texttt{liangyuwei631@gmail.com}}}
\affil{(Mathematics and Applied Mathematics 2302), Zhejiang University }


\date{\today}

\maketitle
\begin{abstract}
    本实验的目的是实现链表数据结构,并测试其功能,包括插入、删除、遍历、拷贝构造和移动构造。测试程序验证了链表的初始化、头尾元素的访问、插入和删除操作、迭代器的正确性、拷贝与赋值操作以及内存管理的正确性。所有测试均通过,未发现内存泄漏。
\end{abstract}

\section{测试程序的设计思路}

为了测试链表的功能,我设计了一个测试程序,涵盖了链表的核心功能。测试思路如下:


\begin{enumerate}
    \item \textbf{链表初始化}:创建一个空的链表对象,并验证其初始状态。
    
    \item \text{头尾元素}:测试 \texttt{front} 和 \texttt{back} 函数,分别返回链表头部和尾部的元素。
    
    \item \textbf{插入操作}:
    \begin{itemize}
        \item 通过 \texttt{push\_front} 和 \texttt{push\_back} 插入元素,分别向链表头部和尾部插入数据。
        \item 验证插入后链表的大小是否更新正确,以及数据的顺序是否符合预期。
    \end{itemize}

    \item \textbf{遍历链表}:
    \begin{itemize}
        \item 使用迭代器遍历链表,输出所有节点数据,测试前置和后置的自增(\texttt{++})和自减(\texttt{--})运算符,确保迭代器可以正确工作。
    \end{itemize}

    \item \textbf{删除操作}:
    \begin{itemize}
        \item 测试 \texttt{pop\_front} 和 \texttt{pop\_back} 删除链表头部和尾部的数据,确保删除后链表的大小和数据符合预期。
    \end{itemize}

    \item \textbf{拷贝与赋值}:
    \begin{itemize}
        \item 测试拷贝构造函数和赋值运算符,将一个链表的数据拷贝到另一个链表。
        \item 使用移动构造函数将链表的数据从一个对象移动到另一个对象,并检查移动后的源链表是否为空。
    \end{itemize}

    \item \textbf{内存检测}:
    \begin{itemize}
        \item 通过 \texttt{valgrind} 工具对测试程序进行内存泄漏检测,确保程序的内存管理是正确的,所有动态分配的内存都被正确释放。
    \end{itemize}
\end{enumerate}

\section{测试的结果}

测试结果一切正常。\\
下面是输出的结果
\begin{lstlisting}[language={}, caption={List.cpp的输出结果}]
    List initialized. Size: 0
    After push_back 10, 20, 30:
    10 20 30
    After push_front 5:
    5 10 20 30
    Testing dynamic iterator ++ and --:
    5 10 20 30
    Using post-increment (it++): 5 10 20 30
    Using pre-decrement (--it):
    30 20 10 5
    Using post-decrement (it--):
    30 20 10 5
    ---
    Front element: 5
    Back element: 30
    After pop_front: 10 20 30
    After pop_back: 10 20
    After push_back 40, 50: 10 20 40 50
    After erasing second element: 10 40 50
    After copying myList to copyList: 10 40 50
    After assigning myList to assignedList: 10 40 50
    After moving assignedList to movedList: 10 40 50
    After moving, is assignedList empty? Yes
    After push_back 60 and push_front 1: 1 10 40 50 60
    After clear. Is myList empty? Yes
\end{lstlisting}

我用 valgrind 进行测试,发现没有发生内存泄露。

\section{bug报告}

\subsection*{访问空链表的 ``front()'' 和 ``back()'' 函数}
当链表为空时,调用 \texttt{front} 和 \texttt{back} 函数会返回默认值。

\subsection*{在空链表上进行 ''erase()'' 操作}
erase() 函数没有检验删除节点的合理性,可能导致头节点或者尾节点被删除的情况。\\
例如当链表为空时,``myList.erase(myList.begin())`` 会抛出异常。

\end{document}

%%% Local Variables: 
%%% mode: latex
%%% TeX-master: t
%%% End: 